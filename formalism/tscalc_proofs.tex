\documentclass{article}

\usepackage{amsmath}
\usepackage{amssymb}
\usepackage{stmaryrd}
\usepackage{listings}
\usepackage{proof}
\usepackage{supertabular}
\usepackage{geometry}
\usepackage{color}
\usepackage{proof}
\usepackage{theorem}

\newtheorem{lem}{Lemma}
\newtheorem{thm}{Theorem}
\newtheorem{defn}{Definition}
\newtheorem{cor}{Corollary}

\newcommand{\qed}{\hfill \mbox{\raggedright \rule{.07in}{.1in}}}

\newenvironment{proof}{\vspace{1ex}\noindent{\bf Proof}\hspace{0.5em}}
  {\hfill\qed\vspace{1ex}}


\input{tscalc_simple}

\begin{document}

\title{TS0 - A simple linear calculus for typestate inference}
\author{Iain McGinniss}

\section{Grammar}

\ottgrammartabular{
\ottt\ottinterrule
\ottv\ottinterrule
\otto\ottinterrule
\ottsv\ottinterrule
\ottT\ottinterrule
\ottst\ottinterrule
\ottO\ottinterrule
}

\section{Operational Semantics}

\ottgrammartabular{
\ottmu\ottinterrule
}

\ottdefnvalidstore

\ottdefnreduce

\section{Type rules}

\ottgrammartabular{
\ottG\ottinterrule
\ottT\ottinterrule
}

\ottdefnvalidgamma

\ottdefnstoretype

\ottdefnmethtype

\ottdefntype

\section{Soundness proofs}

\subsection{Supporting lemmas}

\begin{lem}[Context domains monotonically decrease]
\label{cdom}
If $\Gamma \triangleright t : T \triangleleft \Gamma'$, then
$dom(\Gamma') \subseteq dom(\Gamma)$.
\end{lem}
\begin{proof}
By induction on the typing derivation of $t$. $t$ was typed by one of the
following rules:

\begin{itemize}
\item T-UNIT, T-OBJECT or T-FUN-DEF. It follows
directly from these rules that $\Gamma = \Gamma'$
therefore $dom(\Gamma') \subseteq dom(\Gamma)$.

\item T-LET, therefore $t = \mathtt{let}\:x\:=\:t'\:\mathtt{in}\:t''$
for some $t', t'', \Gamma_1, \Gamma_2, T', T''$ such that
$\Gamma \triangleright t' : T' \triangleleft \Gamma_1$ and
that $\Gamma_1, x : T' \triangleright t'' : T \triangleleft \Gamma', x : T''$
where $\Gamma' = \Gamma_2, x : T''$. By induction, we can derive that
$dom(\Gamma_1) \subseteq dom(\Gamma)$ and that $dom(\Gamma') \subseteq dom(\Gamma_1)$.
Transitively, $dom(\Gamma') \subseteq dom(\Gamma)$.

\item T-UPDATE, therefore $t = x\::=\:t'$
for some $t', \Gamma_1, \Gamma_2, T', T''$ such that
$\Gamma = \Gamma_1, x : T'$,
$\Gamma' = \Gamma_2, x : T''$
and that
$\Gamma_1 \triangleright t' : T'' \triangleleft \Gamma_2$. By induction,
we have that $dom(\Gamma_2) \subseteq dom(\Gamma_1)$. $\Gamma$ and $\Gamma'$ extend
these contexts with the same variable $x$, therefore $dom(\Gamma') \subseteq dom(\Gamma)$.

\item T-FUN-CALL, therefore $t = x ( \overline{x_i} )$.
Directly from T-FUN-CALL, $dom(\Gamma') \subseteq dom(\Gamma)$.

\item T-METH-CALL, therefore $t = x.m$. Directly from T-METH-CALL,
$dom(\Gamma') \subseteq dom(\Gamma)$.

\item T-SEQ, therefore $t = t' ; t''$ for some $t', t'', \Gamma'', T'$ such that
$\Gamma \triangleright t' : T' \triangleleft \Gamma''$ and
$\Gamma'' \triangleright t'' : T \triangleleft \Gamma'$. By induction, we
have $dom(\Gamma'') \subseteq dom(\Gamma)$ and $dom(\Gamma') \subseteq dom(\Gamma'')$.
Transitively, $dom(\Gamma') \subseteq dom(\Gamma)$.

\item T-DROP, therefore for some $\Gamma''$ we can also
type $t$ such that $\Gamma \triangleright t : T \triangleleft \Gamma', \Gamma''$. 
By induction, $dom(\Gamma' \cup \Gamma'') \subseteq \Gamma$,
therefore $dom(\Gamma') \subseteq dom(\Gamma)$.

\end{itemize}
\end{proof}


\begin{lem}[Weakening]
If $\Gamma \triangleright t : T \triangleleft \Gamma'$,
then for all $\Gamma''$ such that $\mathbf{valid} (\Gamma,\Gamma'')$,
$t$ can also be typed such that
$\Gamma, \Gamma'' \triangleright t : T \triangleleft \Gamma', \Gamma''$.
\end{lem}
\begin{proof}
by induction on the structure of $t$.

If $t$ is some value $v$, it follows that the 
$\Gamma = \Gamma'$ and that the typing derivation will still be valid as long
as the input and output contexts are identical. Therefore,
$\Gamma , \Gamma'' \triangleright v : T \triangleleft \Gamma', \Gamma''$.

If $t$ is not a value, then it is of one of the following forms:

\begin{itemize}
\item $t = \mathtt{let}\:x\:=\:t'\:\mathtt{in}\:t''$. It follows from rule T-LET that
there exists $\Gamma_1, \Gamma_2, T', T''$ such that
$\Gamma \triangleright t' : T' \triangleleft \Gamma_1$ and
$\Gamma_1, x : T \triangleright t'' : T \triangleleft \Gamma', x : T''$.
By definition, $\mathbf{valid}(\Gamma,\Gamma'')$ implies $dom(\Gamma) \cap dom(\Gamma'') = \emptyset$.
By Lemma \ref{cdom}, $dom(\Gamma') \subseteq dom(\Gamma_1) \subseteq dom(\Gamma)$.
Therefore, $\mathbf{valid}(\Gamma_1,\Gamma'')$ and $\mathbf{valid}(\Gamma',\Gamma'')$. 
By induction, $t'$ can be typed such
that $\Gamma,\Gamma'' \triangleright t' : T' \triangleleft \Gamma_1,\Gamma''$,
and $t''$ can be typed such that
$\Gamma_1, \Gamma'', x : T \triangleright t'' : T \triangleleft \Gamma', \Gamma'', x : T''$,
as we can assume by Barendregt's convention that $x$ can be made distinct from
any variable names in $\Gamma''$ by relabeling $x$.
Therefore,
$\Gamma, \Gamma'' \triangleright \mathtt{let}\:x\:=\:t'\:\mathtt{in}\:t : T \triangleleft \Gamma', \Gamma''$.

\item $t = x\::=\:t'$. It follows from rule T-UPDATE that
there exists $\Gamma_1, \Gamma_2, T_1, T'$ such that
$\Gamma = \Gamma_1, x : T_1$ and that
$\Gamma' = \Gamma_2, x : T'$, with
$\Gamma_1 \triangleright t' : T' \triangleleft \Gamma_2$. By Lemma \ref{cdom},
$dom(\Gamma_2) \subseteq dom(\Gamma)$, therefore $\mathbf{valid}(\Gamma_2,\Gamma'')$ which
allows by induction for $t'$ to be typed such that
$\Gamma, \Gamma'' \triangleright t' : T_2 \triangleleft \Gamma_2, \Gamma''$.
Therefore $\Gamma, \Gamma'' \triangleright x\::=\:t' : T \triangleleft \Gamma', \Gamma''$.

\item $t = x ( \overline{x_i} )$. It follows from the rule T-FUN-CALL that
there exists a $\Gamma_1$ such that 
$\Gamma = \Gamma_1, x : (\overline{T_i \gg T_i'}) \rightarrow T, \overline{x_i : T_i}$
and that 
$\Gamma' = \Gamma_1, x : (\overline{T_i \gg T_i'}) \rightarrow T, \overline{x_i : T_i'}$.
As $\Gamma_1$ can be arbitrary in the rule T-FUN-CALL, it follows that we
can extend the input and output contexts such that
$\Gamma, \Gamma'' \triangleright x ( \overline{x_i} ) \triangleleft \Gamma', \Gamma''$.

\item $t = x.m$. It follows from the rule T-METH-CALL that there exists
a $\Gamma_1$ such that $\Gamma = \Gamma_1, x : O@S$ and that $\Gamma' = \Gamma_1, x : O@S'$.
As $\Gamma_1$ can be arbitrary in rule T-METH-CALL, it follows that we can
extend the input and out contexts such that
$\Gamma, \Gamma'' \triangleright x.m : T \triangleleft \Gamma', \Gamma''$.

\item $t = t' ; t''$. It follows from the rule T-SEQ that there exists
$\Gamma_1$ and $T'$ such that $\Gamma \triangleright t' : T' \triangleleft \Gamma_1$
and $\Gamma_1 \triangleright t'' : T \triangleleft \Gamma'$. By induction
$\Gamma, \Gamma'' \triangleright t' : T' \triangleleft \Gamma_1, \Gamma''$
as $dom(\Gamma_1) \subseteq dom(\Gamma)$ (Lemma \ref{cdom}) and therefore
$\mathbf{valid}(\Gamma_1,\Gamma'')$.
Similarly, as $dom(\Gamma_1) = dom(\Gamma')$, by induction
$\Gamma_1, \Gamma'' \triangleright t'' : T \triangleleft \Gamma', \Gamma''$.
Therefore $\Gamma, \Gamma'' \triangleright t' : T' \triangleleft \Gamma_1, \Gamma''$.
\end{itemize}

\end{proof}

\begin{lem}[Substitution]
If $\Gamma \triangleright t : T \triangleleft \Gamma'$, then
$\{ \overline{x_i / \rho_i} \} \Gamma \triangleright \{ \overline{x_i / \rho_i} \} t : T \triangleleft \{ \overline{x_i / \rho_i} \} \Gamma'$
where each $x_i$ and $\rho_i$ is distinct, and $\overline{x_i} \cap \overline{\rho_i} = \emptyset$.
\end{lem}
\begin{proof}
By induction on the structure of $t$.

If $t$ is some value $v$, then $\Gamma = \Gamma'$, therefore
$\{ \overline{x_i / \rho_i} \} \Gamma = \{ \overline{x_i / \rho_i} \} \Gamma'$
and substitution has no effect on the value, which has no specific requirements
of either $\Gamma$ or $\Gamma'$ other than that they are equal.
Therefore, $\{ \overline{x_i / \rho_i} \} \Gamma \triangleright \{ \overline{x_i / \rho_i} \} v \triangleleft \{ \overline{x_i / \rho_i} \} \Gamma'$.

It $t$ is a term, then it is of one of the following forms:

\begin{itemize}
\item $t = \mathtt{let}\:x\:=\:t''\:\mathtt{in}\:t'$. 
Substitution is defined on this term such that
$\{ \overline{x_i / \rho_i} \} t \equiv \mathtt{let}\:x\:=\:(\{ \overline{x_i / \rho_i} \} t'')\:\mathtt{in}\:(\{ \overline{x_j / \rho_j} \}t')$, where $\overline{\rho_j} = \overline{\rho_i} / \{ x \}$.

By rule T-LET, it follows
that there exists $T', T'', \Gamma_1$ such that
$\Gamma \triangleright t'' : T'' \triangleleft \Gamma_1$ and that
$\Gamma_1, x : T'' \triangleright t'' : T \triangleleft \Gamma', x : T'$.
Additionally, $x \notin dom(\Gamma)$.

By induction, $t''$ can be substituted such that 
$\{ \overline{x_i / \rho_i} \} \Gamma \triangleright \{ \overline{x_i / \rho_i} \} t'' \triangleleft \{ \overline{x_i / \rho_i} \} \Gamma_1$.
Additionally, $t'$ can be substituted such that
$(\{ \overline{x_j / \rho_j} \} \Gamma_1) , x : T'' \triangleright \{ \overline{x_j / \rho_j} \} t'' \triangleleft \{ (\overline{x_j / \rho_j} \} \Gamma') , x : T'$ as it is guaranteed that $x \notin \overline{\rho_j}$.
As $dom(\Gamma') \subseteq dom(\Gamma_1) \subseteq dom(\Gamma)$ by Lemma \ref{cdom}, and that $x \notin dom(\Gamma)$, it follows that $\{ \overline{x_i / \rho_i} \} \Gamma_1 = (\{ \overline{x_j / \rho_j} \} \Gamma_1)$ and
that $\overline{x_i / \rho_i} \} \Gamma' = \overline{x_j / \rho_j} \} \Gamma'$.

Therefore, the requirements of T-LET are satisfied such that
$\{ \overline{x_i / \rho_i} \} \Gamma \triangleright \{ \overline{x_i / \rho_i} \} t : T \triangleleft \{ \overline{x_i / \rho_i} \} \Gamma'$.

\item $t = \mathtt{x :=\:}t'$. 
Substitution is defined on this term such that
$\{ \overline{x_i / \rho_i} \} ( x\::=\:t' ) \equiv \rho\::=\: \{ \overline{x_i / \rho_i} \} t'$
where $\rho = x$ if $x \notin \overline{\rho_i}$ and $\rho = x_j$ if
$x = \rho_j$ for some $\rho_j \in \overline{\rho_i}$.

By rule T-UPDATE, there exists
$T', T'' \Gamma_1, \Gamma_2$ such that
$\Gamma = \Gamma_1, x : T''$ and $\Gamma' = \Gamma_2, x : T'$ with
$\Gamma_1 \triangleright t' : T'' \triangleleft \Gamma_2$. 

By induction,
substitution can be performed on $t'$ such that
$\{ \overline{x_i / \rho_i} \} \Gamma_1 \triangleright \{ \overline{x_i / \rho_i} \} t' : T'' \triangleleft \{ \overline{x_i / \rho_i} \} \Gamma_2$. 

\begin{itemize}
\item If $x \in \overline{\rho_i}$,
$\{ \overline{x_i / \rho_i} \} \Gamma_1, x : T'' \equiv (\{ \overline{x_i / \rho_i} \} \Gamma_1), x_j : T''$
and
$\{ \overline{x_i / \rho_i} \} \Gamma_2, x : T' \equiv (\{ \overline{x_i / \rho_i} \} \Gamma_2), x_j : T'$.
Therefore, $t$ can by typed by T-UPDATE after the substitution.
\item If $x \notin \overline{\rho_i}$, then
$\{ \overline{x_i / \rho_i} \} \Gamma_1, x : T'' \equiv (\{ \overline{x_i / \rho_i} \} \Gamma_1), x : T''$
and
$\{ \overline{x_i / \rho_i} \} \Gamma_2, x : T' \equiv (\{ \overline{x_i / \rho_i} \} \Gamma_2), x : T'$.
Therefore, $t$ can be typed by T-UPDATE after the substitution.
\end{itemize}

\item $t = t_l ; t_r$.
Substitution is defined on this term such that
$\{ \overline{x_i / \rho_i} \} ( t_l ; t_r ) \equiv \{ \overline{x_i / \rho_i} \} t_l ; \{ \overline{x_i / \rho_i} \} t_r$.

By rule T-SEQ, there exists $\Gamma'', T'$ such that
$\Gamma \triangleright t_l : T' \triangleleft \Gamma''$ and
$\Gamma'' \triangleright t_r : T \triangleleft \Gamma'$. By induction,
$\{ \overline{x_i / \rho_i} \} \Gamma \triangleright \{ \overline{x_i / \rho_i} \} t_l : T' \triangleleft \{ \overline{x_i / \rho_i} \} \Gamma''$ and
$\{ \overline{x_i / \rho_i} \} \Gamma'' \triangleright \{ \overline{x_i / \rho_i} \} t_r : T \triangleleft \{ \overline{x_i / \rho_i} \} \Gamma'$.
Therefore, $t$ can by typed T-SEQ after the substitution.

\item $t = x_0 ( \overline{x_k} )$.
Substition is defined on this term such that
$\{ \overline{x_i / \rho_i} \} x_0 ( \overline{x_k} ) \equiv \rho_0 ( \overline{\rho_k} )$
where for each $x \in \{ x_0, \overline{x_k} \}$,
$x = \rho_j \implies \rho = x_j$, otherwise $\rho$ is the original value.

For each substitution that occurs in the term, the variable must exist in
both the input and output contexts by rule T-FUN-CALL, and will be substituted
for the same name. Therefore, $t$ can be typed by T-FUN-CALL after the
substitution.

\item $t = x.m$.
Substition is defined on this term such that
$\{ \overline{x_i / \rho_i} \} x.m \equiv \rho.m$
where $\rho = x_j$ if $x = \rho_j$ for some $\rho_j \in \overline{\rho_i}$.
Otherwise, $\rho = x$.

Rule T-METH-CALL requires that $x$ exist in both $\Gamma$ and $\Gamma'$ and
therefore any substitution of x in the term will be matched with the
same substitution in the contexts. Therefore, $t$ can be typed by
T-METH-CALL after the substitution.

\end{itemize}
\end{proof}

\subsection{Progress and Preservation}

\begin{thm}[Progress and Preservation]
Given a term $t$ such that $\Gamma \triangleright t : T \triangleleft \Gamma'$
with a store $\mu$ such that $\Gamma \vdash \mu$, either $t$ is a value or 
there exists a $t'$ and $\mu'$ such that
$t \mid \mu \longrightarrow t' \mid \mu'$, with some $\Gamma''$ such that
$\Gamma'' \vdash \mu'$ and $\Gamma'' \triangleright t' : T \triangleleft \Gamma'$.
\end{thm}
\begin{proof}
Proof by induction on the structure of $t$. Assume $t$ is not a value. It is
therefore of one of the following forms:

\begin{itemize}
\item $t = \mathtt{let}\:x\::=\:t''\:\mathtt{in}\:t'$. 
It follows by rule T-LET that there exists $\Gamma_1, T', T''$
such that
$\Gamma \triangleright t'' : T' \triangleleft \Gamma_1$,
$\Gamma_1, x : T' \triangleright t' : T \triangleleft \Gamma', x : T''$.

	\begin{itemize}
	\item If $t''$ is some value $v$, then $\Gamma = \Gamma_1$ and
	reduction can occur
	by R-LET-VALUE, such that $\mathtt{let}\:x\::=\:v\:in\:t' \mid \mu \longrightarrow t' \mid \mu [ x \mapsto v ]$.
	Let $\mu' = \mu [ x \mapsto v ]$. Let $\Gamma'' = \Gamma, x : T'$.
	It follows that that $\Gamma'' \vdash \mu'$, with
	$\Gamma'' \triangleright t' : T \triangleleft \Gamma', x : T'$.
	The requirements of rule T-DROP are satisfied by this, therefore we can 
	type $t'$ such that $\Gamma'' \triangleright t' : T \triangleleft \Gamma'$.

	\item If $t''$ is a term, 
	by induction we have that $t'' \mid \mu \longrightarrow t''' \mid \mu'$, with
	$\Gamma'' \vdash \mu'$ and $\Gamma'' \triangleright t''' : T' \triangleleft \Gamma_1$.
	This satisfies the requirements of R-LET-TERM, meaning $t$ itself can reduce
	such that $\mathtt{let}\:x\::=\:t''\:in\:t' \mid \mu \longrightarrow \mathtt{let}\:x\::=\:t'''\:in\:t' \mid \mu'$.
	Additionally, the conditions of T-LET are satisfied such that
	$\Gamma'' \triangleright \mathtt{let}\:x\::=\:t'''\:in\:t' : T \triangleleft \Gamma'$.

	\end{itemize}

\item $t = \mathtt{x :=\:}t'$. It follows from rule T-UPDATE that $\Gamma = \Gamma_1, x : T$ 
and $\Gamma' = \Gamma_2, x : T'$ with $\Gamma_1 \triangleright t' : T' \triangleleft \Gamma_2$,
and $\Gamma_1, x : T \vdash \mu_1, x \mapsto v$.

	\begin{itemize}
	\item If $t'$ is some value $v'$, 
	then reduction can occur by R-UPDATE-VALUE, such that 
	$x := v' \mid \mu_1, x \mapsto v \longrightarrow unit \mid \mu_1 , x \mapsto v'$.
	Let $\mu' = \mu_1, x \mapsto v'$. Let $\Gamma'' = \Gamma, x : T'$.
	It follows that $\Gamma'' \vdash \mu'$. Additionally, as $t'$ was a value,
	it can be observed that $\Gamma = \Gamma_2$, meaning $\Gamma'' = \Gamma'$
	and therefore
	$\Gamma'' \triangleright unit : Unit \triangleleft \Gamma'$ by rule
	T-UNIT.

	\item If $t'$ is a term, 
	by induction we have that $t' \mid \mu_1 \longrightarrow t'' \mid \mu_2$
	with some $\Gamma_3$ such that $\Gamma_3 \vdash \mu_2$ and
	$\Gamma_3 \triangleright t'' : T' \triangleleft \Gamma_2$.
	This satisfies the requirements of R-UPDATE-TERM, meaning $t$ itself can reduced
	such that $x := t' \mid \mu_1, x \mapsto v \longrightarrow x := t'' \mid \mu_2, x \mapsto v$.
	Let $\mu' = \mu_2, x \mapsto v$. Let $\Gamma''$ = $\Gamma_3, x : T$.
	It follows that $\Gamma'' \vdash \mu'$. 
	Additionally, the requirements of T-UPDATE are satisfied such that
	$\Gamma'' \triangleright x := t'' : Unit \triangleleft \Gamma'$.
	\end{itemize}

\item $t = t_l ; t_r$. It follows by rule
T-SEQ that $\Gamma \triangleright t_l : T_l \triangleleft \Gamma_{mid}$ and
$\Gamma_{mid} \triangleright t_r : T \triangleleft \Gamma'$.

	\begin{itemize}
	\item If $t_l$ is some value $v$, then
	reduction can occur by R-SEQ-LEFT-VALUE such that
	$t_l; t_r \mid \mu \longrightarrow t_r \mu$. Trivially, 
	$\Gamma'' = \Gamma = \Gamma_{mid}$
	and $\mu' = \mu$, therefore $\Gamma'' \vdash \mu'$ and 
	$\Gamma'' \triangleright t_r : T \triangleleft \Gamma'$.

	\item If $t_l$ is a term, by induction $t_l \mid \mu \longrightarrow t_l' \mid \mu'$
	with some $\Gamma'' \vdash \mu'$ and
	$\Gamma'' \triangleright t_l' : T_l \triangleleft \Gamma_{mid}$.
	This satisfies the requirements of R-SEQ-LEFT-TERM
	such that $t_l; t_r \mid \mu \longrightarrow t_l'; t_r \mid \mu'$.
	Additionally, the requirements of T-SEQ are satisfied such that
	$\Gamma'' \triangleright t_l'; t_r : T \triangleleft \Gamma'$.

	\end{itemize}

\item $t = x ( \overline{x_i} )$. It follows by rule T-FUN-CALL that
$\Gamma = \Gamma_1, x : (\overline{Ti \gg Ti'}) \rightarrow T, \overline{x_i : T_i}$ and that
$\Gamma' = \Gamma_1, x : (\overline{Ti \gg Ti'}) \rightarrow T, \overline{x_i : T_i'}$.
As $\Gamma \vdash \mu$, we must have $\mu(x) = \lambda(\overline{\rho_i : T_i \gg T_i'}).t'$
where $\overline{\rho_i : T_i} \triangleright t : T \triangleleft \overline{\rho_i : T_i'}$.
Let $t'' = \{ \overline{x_i / \rho_i} \} t' $.
The term $t$ can be reduced by R-FUN-CALL such that
$x ( \overline{x_i} ) \mid \mu \longrightarrow t'' \mid \mu$.
Let $\mu' = \mu$. Let $\Gamma'' = \Gamma$. By the substitution lemma, 
$\overline{x_i : T_i} \triangleright t'' : T \triangleleft \overline{x_i : T_i'}$.
By applying the weakening lemma, we can type $t''$ such that
$\Gamma'' \triangleright t'' : T \triangleleft \Gamma'$,
with $\Gamma'' \vdash \mu'$.

\item $t = x.m$. It follows by rule T-METH-CALL that
$\Gamma = \Gamma_1, x : O@S$ and that $\Gamma' = \Gamma_1, x : O@S'$. As
$\Gamma \vdash \mu$, we must have $\mu = \mu_1, x \mapsto o@S$ where
$o = [ S \{ m = (v,S') ; ... \} ... ]$ and
$\Gamma_1 \vdash \mu_1$, with 
$\emptyset \triangleright v : T \triangleleft \emptyset$.
Therefore reduction can occur by R-METH-CALL such that
$x.m \mid \mu_1, x \mapsto o@S \longrightarrow v \mid \mu_1, x \mapsto o@S'$.
Let $\mu' = \mu_1, x \mapsto o@S'$. Let $\Gamma'' = \Gamma'$. It follows that
$\Gamma' \vdash \mu'$. Also, as $\Gamma'' = \Gamma'$, it follows by
one of T-UNIT, T-FUN-DEF or T-OBJECT that 
$\Gamma'' \triangleright v : T \triangleleft \Gamma'$.

\end{itemize}

Therefore, well-typed terms with a corresponding well-typed store can always be
reduced, which always produces a new well-typed term with
corresponding well-typed store.

\end{proof}

\section{Type inference}

\ottgrammartabular{
\ottC\ottinterrule
\ottg\ottinterrule
\otttau\ottinterrule
\ottphi\ottinterrule
}

Things we can infer from a typing judgement, assuming it to hold true:

\[
\begin{array}{lll}
\llbracket \Gamma \triangleright \mathtt{unit} : \alpha \triangleleft \Gamma' \rrbracket
& = & 
\alpha = \mathbf{Unit} \wedge \Gamma = \Gamma' \\

\llbracket \Gamma \triangleright [ \overline{S_i \{ \overline{ m_{ij} = (v_{ij}, S_{ij}) } } \} ]@S : \alpha \triangleleft \Gamma' \rrbracket
& = & 
\exists \overline{\alpha_{ij}} .
\alpha = \{ \overline{ S_i \{ m_{ij} : \alpha_{ij} >> S_{ij} \} }\}@S \wedge
\overline{\llbracket \Gamma \triangleright v_{ij} : \alpha_{ij} \triangleleft \Gamma \rrbracket}
\wedge \Gamma = \Gamma' \\

\llbracket \Gamma \triangleright \lambda ( \overline{x_i} ) . t : \alpha \triangleleft \Gamma' \rrbracket
& = &
\exists \Gamma_a, \Gamma_b . \exists \overline{\alpha_{ai}, \alpha_{bi}}, \alpha_r . 
\overline{\Gamma_a(x_i) = \alpha_{ai}} \wedge \overline{\Gamma_b(x_i) = \alpha_{bi}} \\
&&\wedge\:\alpha = (\overline{\alpha_{ai} \gg \alpha_{bi}}) \rightarrow \alpha_r \wedge \Gamma = \Gamma' \\

\llbracket \Gamma \triangleright \mathtt{let}\:x\:\mathtt{=}\:t'\:\mathtt{in}\:t : \alpha \triangleleft \Gamma' \rrbracket
& = & 
\exists \Gamma'', \alpha' . 
\llbracket \Gamma \triangleright t' : \alpha' \triangleleft \Gamma'' \rrbracket \wedge 
\Gamma' = \Gamma'', x : \alpha' \wedge
\alpha = \mathbf{Unit} \\

\llbracket \Gamma \triangleright x\:\mathtt{:=}\:t' : \alpha \triangleleft \Gamma' \rrbracket
& = &
\exists \Gamma'', \alpha' . 
\llbracket \Gamma \triangleright t' : \alpha' \triangleleft \Gamma'' \rrbracket \wedge 
\Gamma' = \Gamma'' [ x \mapsto \alpha' ] \wedge
\alpha = \mathbf{Unit} \\

\llbracket \Gamma \triangleright t ; t' : \alpha \triangleleft \Gamma' \rrbracket
& = &
\exists \Gamma'', \alpha' . 
\llbracket \Gamma \triangleright t : \alpha' \triangleleft \Gamma'' \rrbracket \wedge 
\llbracket \Gamma'' \triangleright t' : \alpha \triangleleft \Gamma' \rrbracket \\

\llbracket \Gamma \triangleright x ( \overline{x_i} ) : \alpha \triangleleft \Gamma' \rrbracket
& = &
\exists \overline{\alpha_i}, \overline{\alpha_i'} .
\overline{\Gamma(x_i) = \alpha_i} \wedge
\Gamma(x) = ( \overline{\alpha_i \gg \alpha_i'} ) \rightarrow \alpha \wedge
\Gamma' = \Gamma [ \overline{x_i \mapsto \alpha_i'} ] \\

\llbracket \Gamma \triangleright x.m : \alpha \triangleleft \Gamma' \rrbracket
& = &
\exists \omega, \sigma, \sigma' . 
\Gamma(x) = \omega @ \sigma \wedge
S.m : \alpha \gg \sigma' \in \omega \wedge
\Gamma' = \Gamma[ x \mapsto \omega @ \sigma' ] \\

\end{array}
\]

\end{document}