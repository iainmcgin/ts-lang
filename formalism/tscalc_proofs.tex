\documentclass{article}

\usepackage{amsmath}
\usepackage{amssymb}
\usepackage{stmaryrd}
\usepackage{listings}
\usepackage{proof}
\usepackage{supertabular}
\usepackage{geometry}
\usepackage{color}
\usepackage{proof}
\usepackage{theorem}

\newtheorem{lem}{Lemma}
\newtheorem{thm}{Theorem}
\newtheorem{defn}{Definition}
\newtheorem{cor}{Corollary}

\newcommand{\Tr}{\mathit{Tr}}
\newcommand{\secref}[1]{Section \ref{#1}}
\newcommand{\figref}[1]{Figure \ref{#1}}
\newcommand{\lstref}[1]{Listing \ref{#1}}
\newcommand{\appref}[1]{Appendix \ref{#1}}
\newcommand{\lemref}[1]{Lemma \ref{#1}}

\newcommand{\rlett}{\textsc{r\_let\_term} }
\newcommand{\rletv}{\textsc{r\_let\_value} }
\newcommand{\rupdt}{\textsc{r\_update\_term} }
\newcommand{\rupdv}{\textsc{r\_update\_value} }
\newcommand{\rseqlt}{\textsc{r\_seq\_left\_term} }
\newcommand{\rseqlv}{\textsc{r\_seq\_left\_value} }
\newcommand{\rfunc}{\textsc{r\_fun\_call} }
\newcommand{\rmethc}{\textsc{r\_meth\_call} }

\newcommand{\tunit}{\textsc{t\_unit} }
\newcommand{\tobj}{\textsc{t\_object} }
\newcommand{\tfundef}{\textsc{t\_fun\_def} }
\newcommand{\tlet}{\textsc{t\_let} }
\newcommand{\tupd}{\textsc{t\_update} }
\newcommand{\tseq}{\textsc{t\_seq} }
\newcommand{\tfunc}{\textsc{t\_fun\_call} }
\newcommand{\tmethc}{\textsc{t\_meth\_call} }
\newcommand{\tdrop}{\textsc{t\_drop} }

\newcommand{\typerule}[4]{#1 \triangleright #2 : #3 \triangleleft #4}
\newcommand{\oprule}[4]{#1 \mid #2\;\longrightarrow\;#3 \mid #4}
\newcommand{\inferrule}[4]{\llbracket #1 \triangleright #2 : #3 \triangleleft #4 \rrbracket}
\newcommand{\inferlhs}[1]{\llbracket \gamma \triangleright #1 : \alpha \triangleleft \gamma' \rrbracket}
\newcommand{\subst}[3]{#3 \{\overline{^{#1}/_{#2}}\}}
\newcommand{\unitv}{\mathtt{unit}}
\newcommand{\unitt}{\mathbf{Unit}}
\newcommand{\funv}[4]{\lambda(\overline{#1 : #2 \gg #3}).#4}
\newcommand{\funt}[3]{(\overline{(#1 \gg #2) \rightarrow #3}}
\newcommand{\lett}[3]{\mathtt{let}\:#1\:\mathtt{=}\:#2\:\mathtt{in}\:#3}
\newcommand{\upt}[2]{#1\:\mathtt{:=}\:#2}
\newcommand{\cand}{\:\wedge\:}
\newcommand{\free}[1]{\mathbf{free}\:#1}
\newcommand{\fresh}[1]{\mathbf{fresh}\:#1}

\newcommand{\qed}{$\blacksquare$}

\newenvironment{proof}{\vspace{1ex}\noindent{\bf Proof}\hspace{0.5em}}
  {\hfill\qed\vspace{1ex}}

\input{tscalc_simple}

\begin{document}

\title{TS0 - A simple linear calculus for typestate inference}
\author{Iain McGinniss}

\section{Grammar}

\ottgrammartabular{
\ottt\ottinterrule
\ottv\ottinterrule
\otto\ottinterrule
\ottsv\ottinterrule
\ottT\ottinterrule
\ottst\ottinterrule
\ottO\ottinterrule
}

\section{Operational Semantics}

\ottgrammartabular{
\ottmu\ottinterrule
}

\ottdefnvalidstore

\ottdefnreduce

\section{Type rules}
\label{sec:typerules}

\ottgrammartabular{
\ottG\ottinterrule
\ottT\ottinterrule
}

\ottdefnvalidgamma

\ottdefnstoretype

\ottdefnmethtype

\ottdefntype

\section{Soundness proofs}

\subsection{Supporting lemmas}

\begin{lem}
\label{lem:valuectx}
If $v$ is a value and $\typerule{\Gamma}{v}{T}{\Gamma'}$, then $\Gamma = \Gamma'$.
\end{lem}
\begin{proof}
Directly from the typing judgements. $v$ is either {\tt unit}, a function
literal or an object literal. Correspondingly, by rules \textsc{t\_unit},
\textsc{t\_fun\_def} and \textsc{t\_object}, $\Gamma = \Gamma'$.
\end{proof}

\begin{lem}[Weakening]
If $\typerule{\Gamma}{t}{T}{\Gamma'}$,
then for all $\Gamma''$ such that $dom(\Gamma) \cap dom(\Gamma) = \emptyset$
$t$ can also be typed such that
$\typerule{\Gamma, \Gamma''}{t}{T}{\Gamma', \Gamma''}$.
\end{lem}
\begin{proof}
by induction on the structure of $t$.

If $t$ is some value $v$, it follows by \lemref{lem:valuectx} that the 
$\Gamma = \Gamma'$ and that the typing derivation will still be valid as long
as the input and output contexts are identical. Therefore,
$\typerule{\Gamma , \Gamma''}{v}{T}{\Gamma', \Gamma''}$.

If $t$ is not a value, then it is of one of the following forms:

\begin{itemize}
\item $t = \lett{x}{t'}{t''}$. It follows from rule \tlet that
there exists $\Gamma_1, \Gamma_2, T', T''$ such that
$\typerule{\Gamma}{t'}{T'}{\Gamma_1}$ and
$\typerule{\Gamma_1, x : T}{t''}{T}{\Gamma', x : T''}$.
By \tlet we have that $dom(\Gamma') = dom(\Gamma)$.
By induction, $t'$ can be typed such
that $\typerule{\Gamma,\Gamma''}{t'}{T'}{\Gamma_1,\Gamma''}$,
and $t''$ can be typed such that
$\typerule{\Gamma_1, \Gamma'', x : T}{t''}{T}{\Gamma', \Gamma'', x : T''}$,
as we can assume by Barendregt's convention that $x$ can be made distinct from
any variable names in $\Gamma''$ by relabeling $x$.
Therefore,
$\typerule{\Gamma, \Gamma''}{\lett{x}{t'}{t}}{T}{\Gamma', \Gamma''}$.

\item $t = x\::=\:t'$. It follows from rule \tupd that
there exists $\Gamma_1, \Gamma_2, T_1, T'$ such that
$\Gamma = \Gamma_1, x : T_1$ and that
$\Gamma' = \Gamma_2, x : T'$, with
$\typerule{\Gamma_1}{t'}{T'}{\Gamma_2}$.
As $\Gamma$ and $\Gamma''$ are disjoint and by \tupd, 
$dom(\Gamma_2) \subseteq dom(\Gamma)$, $t'$ to be typed such that
$\typerule{\Gamma, \Gamma''}{t'}{T_2}{\Gamma_2, \Gamma''}$.
Therefore $\typerule{\Gamma, \Gamma''}{x\::=\:t'}{T}{\Gamma', \Gamma''}$.

\item $t = x ( \overline{x_i} )$. It follows from the rule \tfunc that
there exists a $\Gamma_1$ such that 
$\Gamma = \Gamma_1, x : (\overline{T_i \gg T_i'}) \rightarrow T, \overline{x_i : T_i}$
and that 
$\Gamma' = \Gamma_1, x : (\overline{T_i \gg T_i'}) \rightarrow T, \overline{x_i : T_i'}$.
As $\Gamma_1$ can be arbitrary in the rule \tfunc, it follows that we
can extend the input and output contexts such that
$\typerule{\Gamma, \Gamma''}{x ( \overline{x_i} )}{T}{\Gamma', \Gamma''}$.

\item $t = x.m$. It follows from the rule \tmethc that there exists
a $\Gamma_1$ such that $\Gamma = \Gamma_1, x : O@S$ and that $\Gamma' = \Gamma_1, x : O@S'$.
As $\Gamma_1$ can be arbitrary in rule \tmethc, it follows that we can
extend the input and out contexts such that
$\typerule{\Gamma, \Gamma''}{x.m}{T}{\Gamma', \Gamma''}$.

\item $t = t' ; t''$. It follows from the rule \tseq that there exists
$\Gamma_1$ and $T'$ such that $\typerule{\Gamma}{t'}{T'}{\Gamma_1}$
and $\typerule{\Gamma_1}{t''}{T}{\Gamma'}$. By induction
$\typerule{\Gamma, \Gamma''}{t'}{T'}{\Gamma_1, \Gamma''}$
as $dom(\Gamma_1) \subseteq dom(\Gamma)$ and therefore
$\Gamma_1$ and $\Gamma''$ are disjoint.
Similarly, as $dom(\Gamma_1) = dom(\Gamma')$, by induction
$\typerule{\Gamma_1, \Gamma''}{t''}{T}{\Gamma', \Gamma''}$.
Therefore $\typerule{\Gamma, \Gamma''}{t'}{T'}{\Gamma_1, \Gamma''}$.
\end{itemize}

\end{proof}

\begin{lem}[Substitution]
If $\typerule{\Gamma}{t}{T}{\Gamma'}$, then
$\typerule
{\subst{x_i}{y_i}{\Gamma}}
{\subst{x_i}{y_i}{t}}
{T}
{\subst{x_i}{y_i}{\Gamma'}}$
where each $x_i$ and $y_i$ is distinct, and $\overline{x_i} \cap \overline{y_i} = \emptyset$.
\end{lem}
\begin{proof}
By induction on the structure of $t$.

If $t$ is some value $v$, then $\Gamma = \Gamma'$, therefore
$\subst{x_i}{y_i}{\Gamma} = \subst{x_i}{y_i}{\Gamma'}$
and substitution has no effect on the value, which has no specific requirements
of either $\Gamma$ or $\Gamma'$ other than that they are equal.
Therefore, $\typerule{\subst{x_i}{y_i}{\Gamma}}{\subst{x_i}{y_i}{v}}{T}{\subst{x_i}{y_i}{\Gamma'}}$.

If $t$ is a term, then it is of one of the following forms:

\begin{itemize}
\item $t = \lett{x}{t''}{t'}$. 
Substitution is defined on this term such that
$\subst{x_i}{y_i}{t} \equiv 
\lett{x}{( \subst{x_i}{y_i}{t''} )}{(\{ \overline{x_j / y_j} \}t')}$
, where $\overline{y_j} = \overline{y_i} / \{ x \}$.

By rule \tlet, it follows
that there exists $T', T'', \Gamma_1$ such that
$\typerule{\Gamma}{t''}{T''}{\Gamma_1}$ and that
$\typerule{\Gamma_1, x : T''}{t''}{T}{\Gamma', x : T'}$.
Additionally, $x \notin dom(\Gamma)$.

By induction, $t''$ can be substituted such that 
$\typerule
{\subst{x_i}{y_i}{\Gamma}}
{\subst{x_i}{y_i}{t''}}
{\subst{x_i}{y_i}{\Gamma_1}}
$.
Additionally, $t'$ can be substituted such that
$(\subst{x_j}{y_j}{\Gamma_1}) , x : T'' \triangleright \subst{x_j}{y_j}{t''} \triangleleft \{ (\overline{x_j / y_j} \} \Gamma') , x : T'$ as it is guaranteed that $x \notin \overline{y_j}$.
As $dom(\Gamma') = dom(\Gamma)$, and that $x \notin dom(\Gamma)$, it follows that $\subst{x_i}{y_i}{\Gamma_1} = (\subst{x_j}{y_j}{\Gamma_1})$ and
that $\overline{x_i / y_i} \} \Gamma' = \overline{x_j / y_j} \} \Gamma'$.

Therefore, the requirements of \tlet are satisfied such that
$\typerule{\subst{x_i}{y_i}{\Gamma}}{\subst{x_i}{y_i}{t}}{T}{\subst{x_i}{y_i}{\Gamma'}}$.

\item $t = \mathtt{x :=\:}t'$. 
Substitution is defined on this term such that
$\{ \overline{x_i / y_i} \} ( x\::=\:t' ) \equiv y\::=\: \subst{x_i}{y_i}{t'}$
where $y = x$ if $x \notin \overline{y_i}$ and $y = x_j$ if
$x = y_j$ for some $y_j \in \overline{y_i}$.

By rule \tupd, there exists
$T', T'' \Gamma_1, \Gamma_2$ such that
$\Gamma = \Gamma_1, x : T''$ and $\Gamma' = \Gamma_2, x : T'$ with
$\typerule{\Gamma_1}{t'}{T''}{\Gamma_2}$. 

By induction,
substitution can be performed on $t'$ such that
$\typerule{\subst{x_i}{y_i}{\Gamma_1}}{\subst{x_i}{y_i}{t'}}{T''}{\subst{x_i}{y_i}{\Gamma_2}}$. 

\begin{itemize}
\item If $x \in \overline{y_i}$,
$\subst{x_i}{y_i}{\Gamma_1}, x : T'' \equiv (\subst{x_i}{y_i}{\Gamma_1}), x_j : T''$
and
$\subst{x_i}{y_i}{\Gamma_2}, x : T' \equiv (\subst{x_i}{y_i}{\Gamma_2}), x_j : T'$.
Therefore, $t$ can by typed by \tupd after the substitution.
\item If $x \notin \overline{y_i}$, then
$\subst{x_i}{y_i}{\Gamma_1}, x : T'' \equiv (\subst{x_i}{y_i}{\Gamma_1}), x : T''$
and
$\subst{x_i}{y_i}{\Gamma_2}, x : T' \equiv (\subst{x_i}{y_i}{\Gamma_2}), x : T'$.
Therefore, $t$ can be typed by \tupd after the substitution.
\end{itemize}

\item $t = t_l ; t_r$.
Substitution is defined on this term such that
$\{ \overline{x_i / y_i} \} ( t_l ; t_r ) \equiv \subst{x_i}{y_i}{t_l} ; \subst{x_i}{y_i}{t_r}$.

By rule \tseq, there exists $\Gamma'', T'$ such that
$\typerule{\Gamma}{t_l}{T'}{\Gamma''}$ and
$\typerule{\Gamma''}{t_r}{T}{\Gamma'}$. By induction,
$\typerule{\subst{x_i}{y_i}{\Gamma}}{\subst{x_i}{y_i}{t_l}}{T'}{\subst{x_i}{y_i}{\Gamma''}}$ and
$\typerule{\subst{x_i}{y_i}{\Gamma''}}{\subst{x_i}{y_i}{t_r}}{T}{\subst{x_i}{y_i}{\Gamma'}}$.
Therefore, $t$ can by typed \tseq after the substitution.

\item $t = x_0 ( \overline{x_k} )$.
Substition is defined on this term such that
$\subst{x_i}{y_i}{x_0} ( \overline{x_k} ) \equiv y_0 ( \overline{y_k} )$
where for each $x \in \{ x_0, \overline{x_k} \}$,
$x = y_j \implies y = x_j$, otherwise $y$ is the original value.

For each substitution that occurs in the term, the variable must exist in
both the input and output contexts by rule \tfunc, and will be substituted
for the same name. Therefore, $t$ can be typed by \tfunc after the
substitution.

\item $t = x.m$.
Substition is defined on this term such that
$\subst{x_i}{y_i}{x.m} \equiv y.m$
where $y = x_j$ if $x = y_j$ for some $y_j \in \overline{y_i}$.
Otherwise, $y = x$.

Rule \tmethc requires that $x$ exist in both $\Gamma$ and $\Gamma'$ and
therefore any substitution of x in the term will be matched with the
same substitution in the contexts. Therefore, $t$ can be typed by
\tmethc after the substitution.

\end{itemize}
\end{proof}

\subsection{Progress and Preservation}

\begin{thm}[Progress and Preservation]
Given a term $t$ such that $\typerule{\Gamma}{t}{T}{\Gamma'}$
with a store $\mu$ such that $\Gamma \vdash \mu$, either $t$ is a value or 
there exist $t'$ and $\mu'$ such that
$\oprule{t}{\mu}{t'}{\mu'}$, with some $\Gamma''$ such that
$\Gamma'' \vdash \mu'$ and $\typerule{\Gamma''}{t'}{T}{\Gamma'}$.
\end{thm}

\begin{proof}
by induction on the structure of $t$. Assume $t$ is not a value. It is
therefore of one of the following forms:

\begin{itemize}
\item $t = \lett{x}{t''}{t'}$. 
It follows by rule \tlet that there exist $\Gamma_1, T', T''$
such that
$\typerule{\Gamma}{t''}{T'}{\Gamma_1}$ and that
$\typerule{\Gamma_1, x : T'}{t'}{T}{\Gamma', x : T''}$. There are two
possibilities:

	\begin{itemize}
	\item If $t''$ is some value $v$, then by \lemref{lem:valuectx} 
	we have that $\Gamma = \Gamma_1$ and reduction can occur
	by \rletv, such that 
	$\oprule{\lett{x}{v}{t'}}{\mu}{t'}{\mu [ x \mapsto v ]}$.
	Let $\mu' = \mu [ x \mapsto v ]$. Let $\Gamma'' = \Gamma, x : T'$.
	It follows that that $\Gamma'' \vdash \mu'$, with
	$\typerule{\Gamma''}{t'}{T}{\Gamma', x : T'}$.
	The requirements of rule \tdrop are satisfied by this, therefore we can 
	type $t'$ such that $\typerule{\Gamma''}{t'}{T}{\Gamma'}$.

	\item If $t''$ is a term, 
	by induction we have that $\oprule{t''}{\mu}{t'''}{\mu'}$, with
	$\Gamma'' \vdash \mu'$ and $\typerule{\Gamma''}{t'''}{T'}{\Gamma_1}$.
	This satisfies the requirements of \rlett, meaning $t$ itself can reduce
	such that $\oprule{\lett{x}{t''}{t'}}{\mu}{\lett{x}{t'''}{t'}}{\mu'}$.
	Additionally, the conditions of \tlet are satisfied such that
	$\typerule{\Gamma''}{\lett{x}{t'''}{t'}}{T}{\Gamma'}$.

	\end{itemize}

\item $t = \mathtt{x :=\:}t'$. It follows from rule \tupd that $\Gamma = \Gamma_1, x : T$ 
and $\Gamma' = \Gamma_2, x : T'$ with $\typerule{\Gamma_1}{t'}{T'}{\Gamma_2}$,
and $\Gamma_1, x : T \vdash \mu_1, x \mapsto v$.

	\begin{itemize}
	\item If $t'$ is some value $v'$, 
	then reduction can occur by \\
	\rupdv, such that 
	$x := v' \mid \mu_1, x \mapsto v \longrightarrow \unitv \mid \mu_1 , x \mapsto v'$.
	Let $\mu' = \mu_1, x \mapsto v'$. Let $\Gamma'' = \Gamma, x : T'$.
	It follows that $\Gamma'' \vdash \mu'$. Additionally, as $t'$ was a value,
	it can be observed that $\Gamma = \Gamma_2$, meaning $\Gamma'' = \Gamma'$
	and therefore
	$\typerule{\Gamma''}{\unitv}{\unitt}{\Gamma'}$ by rule
	\tunit.

	\item If $t'$ is a term, 
	by induction we have that $t' \mid \mu_1 \longrightarrow t'' \mid \mu_2$
	with some $\Gamma_3$ such that $\Gamma_3 \vdash \mu_2$ and
	$\typerule{\Gamma_3}{t''}{T'}{\Gamma_2}$.
	This satisfies the requirements of \rupdt, meaning $t$ itself can reduced
	such that $x := t' \mid \mu_1, x \mapsto v \longrightarrow x := t'' \mid \mu_2, x \mapsto v$.
	Let $\mu' = \mu_2, x \mapsto v$. Let $\Gamma''$ = $\Gamma_3, x : T$.
	It follows that $\Gamma'' \vdash \mu'$. 
	Additionally, the requirements of \tupd are satisfied such that
	$\typerule{\Gamma''}{x := t''}{\unitt}{\Gamma'}$.
	\end{itemize}

\item $t = t_l ; t_r$. It follows by rule
\tseq that $\typerule{\Gamma}{t_l}{T_l}{\Gamma_{mid}}$ and
$\typerule{\Gamma_{mid}}{t_r}{T}{\Gamma'}$.

	\begin{itemize}
	\item If $t_l$ is some value $v$, then
	reduction can occur by \\ 
	\rseqlv such that
	$t_l; t_r \mid \mu \longrightarrow t_r \mu$. By \lemref{lem:valuectx}, 
	$\Gamma'' = \Gamma = \Gamma_{mid}$
	and $\mu' = \mu$, therefore $\Gamma'' \vdash \mu'$ and 
	$\typerule{\Gamma''}{t_r}{T}{\Gamma'}$.

	\item If $t_l$ is a term, by induction $t_l \mid \mu \longrightarrow t_l' \mid \mu'$
	with some $\Gamma'' \vdash \mu'$ and
	$\typerule{\Gamma''}{t_l'}{T_l}{\Gamma_{mid}}$.
	This satisfies the requirements of \rseqlt
	such that $t_l; t_r \mid \mu \longrightarrow t_l'; t_r \mid \mu'$.
	Additionally, the requirements of \tseq are satisfied such that
	$\typerule{\Gamma''}{t_l'; t_r}{T}{\Gamma'}$.

	\end{itemize}

\item $t = x ( \overline{x_i} )$. It follows by rule \tfunc that
$\Gamma = \Gamma_1, x : (\overline{Ti \gg Ti'}) \rightarrow T, \overline{x_i : T_i}$ and that
$\Gamma' = \Gamma_1, x : (\overline{Ti \gg Ti'}) \rightarrow T, \overline{x_i : T_i'}$.
As $\Gamma \vdash \mu$, we must have $\mu(x) = \lambda(\overline{y_i : T_i \gg T_i'}).t'$
where $\typerule{\overline{y_i : T_i}}{t}{T}{\overline{y_i : T_i'}}$.
Let $t'' = \subst{x_i}{y_i}{t'} $.
The term $t$ can be reduced by \rfunc such that
$x ( \overline{x_i} ) \mid \mu \longrightarrow t'' \mid \mu$.
Let $\mu' = \mu$. Let $\Gamma'' = \Gamma$. By the substitution lemma, 
$\typerule{\overline{x_i : T_i}}{t''}{T}{\overline{x_i : T_i'}}$.
By applying the weakening lemma, we can type $t''$ such that
$\typerule{\Gamma''}{t''}{T}{\Gamma'}$,
with $\Gamma'' \vdash \mu'$.

\item $t = x.m$. It follows by rule \tmethc that
$\Gamma = \Gamma_1, x : O@S$ and that $\Gamma' = \Gamma_1, x : O@S'$. As
$\Gamma \vdash \mu$, we must have $\mu = \mu_1, x \mapsto o@S$ where
$o = [ S \{ m = (v,S') ; ... \} ... ]$ and
$\Gamma_1 \vdash \mu_1$, with 
$\typerule{\emptyset}{v}{T}{\emptyset}$.
Therefore reduction can occur by \rmethc such that
$x.m \mid \mu_1, x \mapsto o@S \longrightarrow v \mid \mu_1, x \mapsto o@S'$.
Let $\mu' = \mu_1, x \mapsto o@S'$. Let $\Gamma'' = \Gamma'$. It follows that
$\Gamma' \vdash \mu'$. Also, as $\Gamma'' = \Gamma'$, it follows by
one of \tunit, \tfundef or \tobj that 
$\typerule{\Gamma''}{v}{T}{\Gamma'}$.
\end{itemize}
\end{proof}

\section{Type inference}

Type inference in the style of Pottier, i.e. separate constraint generation
and constraint solving phases to determine the principal typing of a term,
can be defined by observing the requirements imposed by the typing judgements
defined in Section \ref{sec:typerules}.

\ottgrammartabular{
\ottC\ottinterrule
\ottctxc\ottinterrule
\otttyc\ottinterrule
\ottprotoc\ottinterrule
}

\[
\begin{array}{lll}

% unit value
\inferlhs{\unitv}
& = & 
\alpha = \unitt\:\wedge 
\\ && 
\Gamma = \Gamma' 
\\
\\

% object value
\inferlhs{[ \overline{S_i \{ \overline{ m_{ij} = (v_{ij}, S_{ij}) } } \} ]@S}
& = & 
\exists \overline{\alpha_{ij}} .
\\ && 
\alpha = \{ \overline{ S_i \{ m_{ij} : \alpha_{ij} >> S_{ij} \} }\}@S\:\wedge
\\ && 
\Gamma = \Gamma'\:\wedge
\\ && 
\overline{\inferrule{\Gamma}{\triangleright v_{ij}}{\alpha_{ij}}{\Gamma}}
\\
\\

% function value
\inferlhs{\lambda ( \overline{x_i} ) . t}
& = &
\exists \Gamma_a, \Gamma_b . \overline{\alpha_{ai}, \alpha_{bi}}, \alpha_r . 
\\ && 
\Gamma_a = \overline{x_i : \alpha_{ai}}\:\wedge 
\\ && 
\Gamma_b = \overline{x_i : \alpha_{bi}}\:\wedge
\\ && 
\alpha = (\overline{\alpha_{ai} \gg \alpha_{bi}}) \rightarrow \alpha_r\:\wedge
\\ && 
\Gamma = \Gamma'\:\wedge
\\ && 
\inferrule{\Gamma_a}{t}{\alpha_r}{\Gamma_b} 
\\
\\

% let binding
\inferlhs{\lett{x}{t'}{t}}
& = & 
\exists \Gamma_v, \Gamma_{b1}, \Gamma_{b2}, \alpha' . 
\\ && 
\Gamma_{b1} = \Gamma_v, x : \alpha'\:\wedge
\\ &&
\Gamma' = \Gamma_{b2} / \{ x \}
\\ &&
\inferrule{\Gamma}{t'}{\alpha'}{\Gamma_v}\:\wedge
\\ && 
\inferrule{\Gamma_{b1}}{t}{\alpha}{\Gamma_{b2}}
\\ && 
\\

% update
\inferlhs{x\:\mathtt{:=}\:t'}
& = &
\exists \Gamma_a, \Gamma_b, \alpha' .
\\ &&
\Gamma_a = \Gamma / \{ x \}\:\wedge
\\ && 
\Gamma' = \Gamma_b, x \mapsto \alpha'\:\wedge
\\ && 
\alpha = \unitt\:\wedge
\\ && 
\inferrule{\Gamma_a}{t'}{\alpha'}{\Gamma_b}
\\
\\

% sequence
\inferlhs{t ; t'}
& = &
\exists \Gamma'', \alpha' . 
\\ && 
\inferrule{\Gamma}{t}{\alpha'}{\Gamma''}\:\wedge
\\ && 
\inferrule{\Gamma''}{t'}{\alpha}{\Gamma'}
\\
\\

% function call
\inferlhs{x ( \overline{x_i} )}
& = &
\exists \overline{\alpha_i}, \overline{\alpha_i'} .
\\ && 
\overline{\Gamma(x_i) = \alpha_i}\:\wedge
\\ && 
\Gamma(x) = ( \overline{\alpha_i \gg \alpha_i'} ) \rightarrow \alpha\:\wedge
\\ && 
\Gamma' = \Gamma [ \overline{x_i \mapsto \alpha_i'} ]
\\
\\

% method call
\inferlhs{x.m}
& = &
\exists \omega, \sigma, \sigma' . 
\\ && 
\Gamma(x) = \omega @ \sigma\:\wedge
\\ && 
m : \alpha \gg \sigma' \in \omega @ \sigma\:\wedge
\\ && 
\Gamma' = \Gamma[ x \mapsto \omega @ \sigma' ]
\\
\\

\end{array}
\]

A set of constraints generated following these rules can then be solved by
using the following strategy:

\begin{itemize}
\item By substitution, determine the variable to type variable mappings
contained in all contexts.
\item Using these fully explicit context definitions, generate additional
type variable constraints by comparing the context variable constraints
to the known type variable mapping for the referenced context.
\item At this point, we have all the information needed to perform first
order unification of all type variable constraints.
\item Finally, we will have a set of method constraints related to each
object. From this, we can construct the explicit finite state machine
for each object variable.
\end{itemize}

\end{document}